\documentclass[12pt, a4paper]{article}

%-----------------------------------------------------------------------
% Preamble by the organize commitee

\usepackage[utf8]{inputenc}
\usepackage[brazil]{babel}
\usepackage[margin=2.5cm]{geometry}
\usepackage{setspace}
\usepackage{indentfirst}
\usepackage{graphicx}
\usepackage{xcolor}
\usepackage{fancyhdr}
\usepackage{url}
\usepackage{enumerate}
\usepackage{amsmath, amsthm, amsfonts, amssymb, amsxtra}
\usepackage{bm}

\pagestyle{fancy}
\fancyhf{}
\lhead{$63^{\textrm{a}}$ RBras}
\rhead{23 a 25 de maio de 2018, Curitiba - PR}
% \cfoot{\thepage}
\renewcommand{\headrulewidth}{0.4pt}
\addtolength{\headheight}{12.0pt}

\makeatletter
\def\@xfootnote[#1]{%
  \protected@xdef\@thefnmark{#1}%
  \@footnotemark\@footnotetext}
\makeatother

\usepackage{hyperref}
\definecolor{mycol}{rgb}{0.0, 0.0, 0.5}
\urlstyle{tt}
\makeatletter
\hypersetup{
  pdftitle={\@title},
  pdfauthor={\@author},
  colorlinks=true,
  linkcolor=mycol,
  citecolor=mycol,
  filecolor=mycol,
  urlcolor=mycol,
  bookmarksdepth=4
}
\makeatother

%-----------------------------------------------------------------------
% Init the document

\begin{document}
\onehalfspacing

%-------------------------------------------
% Título
\begin{center}
  \textbf{
    \Large{Estratégias para análise de contagens sub e
      superdispersas}} \\[1em]
\end{center}

%-------------------------------------------
% Autores
\begin{flushright}
  {\bf Eduardo Elias Ribeiro Junior}
  \footnote[$\dagger$]{Contato:
    \href{mailto:jreduardo@usp.br}{\tt jreduardo@usp.br}}
  \footnote[1]{Departamento de Ciências Exatas (LCE) - ESALQ-USP}
  \footnote[2]{Laboratório de Estatística e Geoinformação (LEG) -
    UFPR}\\
  {\bf Clarice Garcia Borges Demétrio} \footnotemark[1]
\end{flushright}

\vspace*{0.5cm}

\noindent Na análise de dados em forma de contagens, comumente, a
suposição de equidispersão não é adequada e, consequentemente, os
modelos de regressão Poisson são inapropriados. Importantes avanços na
área de análise de contagens têm sido relatados na literatura,
principalmente, para modelar diferentes níveis de dispersão,
nomeadamente, sub (média $>$ variância) e superdispersão (média $<$
variância). Neste artigo, são revisados os modelos COM-Poisson,
\textit{Gamma-Count}, Poisson generalizada e Poisson-Tweedie. A gênesis
de cada modelo é apresentada, juntamente como um resumo comparativo das
distribuições. A similaridade dos modelos COM-Poisson e
\textit{Gamma-Count} é destacada, assim como a flexibilidade dos modelos
Poisson generalizado e Poisson-Tweedie para modelar superdispersão. A
aplicações dos modelos é ilustrada com a análise do número de progênies
de \textit{Sitophilus zeamais}, observado em um experimente completamente
casualizado com quatro subestratos de milho como tratamentos e dez
repetições. A implementação computacional é realizada no sofware
\texttt{R}, cujos códigos são disponibilizadas em material
suplementar.\\

\noindent{\bf Palavras-chave}:
{\it Distribuição COM-Poisson, Distribuição Gamma-Count, Distribuição
  Família Poisson-Tweedie, Poisson generalizada, Superdispersão,
  Subdispersão}.\\

\end{document}